%% This document created by Scientific Word (R) Version 3.0

\documentclass{article}
\usepackage{graphicx}
\usepackage{amsmath}
\usepackage{amsfonts}
\usepackage{amssymb}
%TCIDATA{OutputFilter=latex2.dll}
%TCIDATA{CSTFile=LaTeX article (bright).cst}
%TCIDATA{Created=Tue Sep 16 14:05:51 2003}
%TCIDATA{LastRevised=Tue Sep 16 17:35:15 2003}
%TCIDATA{<META NAME="GraphicsSave" CONTENT="32">}
%TCIDATA{<META NAME="DocumentShell" CONTENT="General\Blank Document">}
%TCIDATA{Language=American English}
\newtheorem{theorem}{Theorem}
\newtheorem{acknowledgement}[theorem]{Acknowledgement}
\newtheorem{algorithm}[theorem]{Algorithm}
\newtheorem{assumption}[theorem]{Assumption}
\newtheorem{axiom}[theorem]{Axiom}
\newtheorem{case}[theorem]{Case}
\newtheorem{claim}[theorem]{Claim}
\newtheorem{conclusion}[theorem]{Conclusion}
\newtheorem{condition}[theorem]{Condition}
\newtheorem{conjecture}[theorem]{Conjecture}
\newtheorem{corollary}[theorem]{Corollary}
\newtheorem{criterion}[theorem]{Criterion}
\newtheorem{definition}[theorem]{Definition}
\newtheorem{example}[theorem]{Example}
\newtheorem{exercise}[theorem]{Exercise}
\newtheorem{lemma}[theorem]{Lemma}
\newtheorem{notation}[theorem]{Notation}
\newtheorem{problem}[theorem]{Problem}
\newtheorem{proposition}[theorem]{Proposition}
\newtheorem{remark}[theorem]{Remark}
\newtheorem{solution}[theorem]{Solution}
\newtheorem{summary}[theorem]{Summary}
\newenvironment{proof}[1][Proof]{\textbf{#1.} }{\ \rule{0.5em}{0.5em}}

\begin{document}
\title{How to Characterize Solutions to Constrained Optimization Problems}
\author{Michael Peters}
\date{\today}
\maketitle
\section{Introduction}

A common technique for characterizing maximum and minimum points in math is to
use first order conditions. When a function reaches its maximum, its
derivative must be zero. The zero derivative can often be interpreted. When
the maximization is subject to an \emph{equality} constraint, the problem
isn't much harder (see the special case below). However, in economics,
maximization typically occurs subject to various inequality constraints. For
example, demand for a particular commodity can't be negative (though it could
be zero). Labour supply can't exceed 24 hours, but will only exceptionally be
equal to 24 hours. 

To deal with this, use the following theorem: Let $f\left(  x_{1},\dots
x_{n}\right)  $ be a real valued function with $n$ real arguments. Let
$c_{1}-G_{1}\left(  x_{1},\dots x_{n}\right)  \leq0$, $\dots c_{m}%
-G_{m}\left(  x_{1},\dots x_{n}\right)  $ be a series of $m$ constraints,
where each of the $G_{i}\left(  \cdot\right)  $ functions is a real valued
function. Let $x^{\ast}=\left\{  x_{1}^{\ast},\dots x_{n}^{\ast}\right\}  $ be
a solution to the problem%
\[
\max f\left(  x_{1},\dots x_{n}\right)
\]
subject to%
\[
c_{1}-G_{1}\left(  x_{1},\dots x_{n}\right)  \leq0
\]%
\[
\vdots
\]%
\[
c_{m}-G_{m}\left(  x_{1},\dots x_{n}\right)  \leq0
\]
and suppose that the functions $G_{1}$ through $G_{m}$ satisfy the
\emph{constraint qualification}\footnote{The constraint qualification says
that the vectors $\left[  \partial G_{i}\left(  x^{\ast}\right)  /\partial
x_{1},\dots\partial G_{i}\left(  x^{\ast}\right)  /\partial x_{n}\right]  $
for each $i$ for which the constraint holds with equality in the optimal
solution are linearly independent.}. Then there is a set of $m$ constants
$\lambda^{\ast}=\left\{  \lambda_{1}^{\ast},\dots\lambda_{m}^{\ast}\right\}  $
such that%
\[
L\left[  x_{1},\dots x_{n},\lambda_{1},\dots\lambda_{m}\right]  \equiv
f\left(  x_{1},\dots x_{n}\right)  +\sum_{i=1}^{m}\lambda_{i}[c_{i}%
-G_{i}\left(  x_{1},\dots x_{n}\right)  ]
\]
and%
\begin{equation}
\frac{\partial L\left[  x_{1}^{\ast},\dots x_{n}^{\ast},\lambda_{1}^{\ast
},\dots\lambda_{m}^{\ast}\right]  }{\partial x_{j}}=0\label{with_x}%
\end{equation}
for $j=1,\dots n$ and%
\begin{equation}
\frac{\partial L\left[  x_{1}^{\ast},\dots x_{n}^{\ast},\lambda_{1}^{\ast
},\dots\lambda_{m}^{\ast}\right]  }{\partial\lambda_{i}}\leq0;\lambda_{i}%
\leq0\label{with_lambda}%
\end{equation}
for all $i$ with complementary slackness.\footnote{Complementary slackness
means that%
\[
\lambda_{i}\cdot\frac{\partial L\left[  x_{1}^{\ast},\dots\lambda_{m}^{\ast
}\right]  }{\partial\lambda_{i}}=0
\]
for all $i$.}

The idea is roughly that once you have found a solution, it might be that you
are prevented from getting the outcome you would like by one of the
constraints. The derivative of your objective function won't be zero. If that
is so, then it should be possible to calculate the \emph{marginal }gain you
could attain by violating that constraint by a little bit and imposing a
penalty for marginal violations (that is what the $\lambda_{j}$'s are for).
The \emph{Lagrangian} function $L\left[  \cdot,\cdot\right]  $ is the sum of
your main objective and all these penalties. At your optimal $x^{\ast}$ the
derivative of this Lagrangian function could be made to be exactly zero if you
could impose penalties that exactly offset the marginal gains of violating the constraints.

Notice that since the constraints are of the form $c_{i}-G_{i}\left(
x_{1},\dots x_{m}\right)  \leq0$ we want the penalty to be positive whenever
$c_{i}-G_{i}\left(  x_{1},\dots x_{m}\right)  >0$. That is why we want the 'fines'
for violating the constraint to be negative in (\ref{with_lambda}). This also
explains complementary slackness. Since the Lagrangian function is linear in
the $\lambda_{j}$, the derivative $\frac{\partial L}{\partial\lambda_{i}}$
will always be equal to the value on the left hand side of one of the
constraints. If this value is strictly negative at the solution, we don't want
to impose a penalty because that would force the decision maker away from the
right solution. As a consequence, we need to have the penalty equal to zero.
On the other hand, if the constraint were exactly satisfied at the optimal
solution, then we would typically be able to do strictly better by violating
the constraint and the penalty would need to be positive.

\subsection{Some simple examples}

The function $-\left(  x^{2}-2x\right)  $ has a maximum point at $x=1$ where
the function takes on a value equal to $1$. This is illustrated in Figure $1$. %

%TCIMACRO{\FRAME{ftbpF}{2.2148in}{1.8922in}{0pt}{}{}{lagrangian_note_fig1.eps}%
%{\special{ language "Scientific Word";  type "GRAPHIC";
%maintain-aspect-ratio TRUE;  display "USEDEF";  valid_file "F";
%width 2.2148in;  height 1.8922in;  depth 0pt;  original-width 2.1101in;
%original-height 1.7997in;  cropleft "0";  croptop "1";  cropright "1";
%cropbottom "0";  filename 'lagrangian_note_fig1.eps';file-properties "XNPEU";}%
%}}%
%BeginExpansion
\begin{figure}
[ptb]
\begin{center}
\includegraphics[
height=1.8922in,
width=2.2148in
]%
{lagrangian_note_fig1.eps}%
\end{center}
\end{figure}
%EndExpansion
Suppose we want to maximize the function $-\left(  x^{2}-2x\right)  $ subject
to the constraint that $0\leq x\leq\frac{1}{2}$. The solution $x^{\ast}$ is
obviously equal to $\frac{1}{2}$ as shown in the picture. At this point the
derivative of $-\left(  x^{2}-2x\right)  $ is equal to $1$. So if we were to
raise $x$ and violate the constraint by a little bit, we would gain $1$. To
prevent this we would intuitively like to create a penalty that imposes a cost
$1$ for each unit by which the constraint is violated. Then on the margin,
increasing $x$ a little would cause the objective to rise by $1$, but the
penalty cost would rise by exactly the same amount - the derivative of the
Lagrangian function would be exactly zero. Then we could think about solutions
to constrained optimization problems by looking at first order conditions
exactly as in unconstrained problems.

We could put this into the form needed by the theorem. We want to maximize%
\[
f\left(  x\right)  =-\left(  x^{2}-2x\right)
\]
There are two constraints, $x$ needs to be at least zero and no more than
$\frac{1}{2}$. We need to write them both as less than or equal to zero
constraints, i.e.,%
\[
-x\leq0
\]
and%
\[
x-\frac{1}{2}\leq0
\]
so $c_{1}=0$ and $G_{1}\left(  x\right)  =x$, while $c_{2}=-\frac{1}{2}$ and
$G_{2}\left(  x\right)  =-x$. 

The theorem now says that when $x=x^{\ast}=\frac{1}{2}$ we will be able to
find a pair of penalties $\lambda_{1}^{\ast}$ and $\lambda_{2}^{\ast}$ such
that if we define the function
\[
L\left[  x,\lambda_{1},\lambda_{2}\right]  =-\left(  x^{2}-2x\right)
+\lambda_{1}\left[  -x\right]  +\lambda_{2}\left[  -\frac{1}{2}-\left(
-x\right)  \right]
\]
then%
\[
\left.  \frac{\partial L\left[  x,\lambda_{1},\lambda_{2}\right]  }{\partial
x}\right|  _{x=\frac{1}{2},\lambda_{1}=\lambda_{1}^{\ast},\lambda_{2}%
=\lambda_{2}^{\ast}}=0
\]%
\[
\left.  \frac{\partial L\left[  x,\lambda_{1},\lambda_{2}\right]  }%
{\partial\lambda_{1}}\right|  _{x=\frac{1}{2},\lambda_{1}=\lambda_{1}^{\ast
},\lambda_{2}=\lambda_{2}^{\ast}}\leq0;\lambda_{1}^{\ast}\leq0
\]
and%
\[
\left.  \frac{\partial L\left[  x,\lambda_{1},\lambda_{2}\right]  }%
{\partial\lambda_{2}}\right|  _{x=\frac{1}{2},\lambda_{1}=\lambda_{1}^{\ast
},\lambda_{2}=\lambda_{2}^{\ast}}\leq0;\lambda_{2}^{\ast}\leq0
\]
where the last two conditions both hold with complementary slackness.

So could we characterize $x^{\ast}=\frac{1}{2}$ using a first order condition?
The derivative of the objective is positive at $\frac{1}{2}$ so we need a
penalty, and since the derivative of objective is $1$, we should intuitively
set this penalty to $1$. We could accomplish this by setting $\lambda
_{2}^{\ast}=-1$. At the optimal solutions we don't need to impose any
penalties for the constraint at zero, because it is not \emph{binding} at the
optimal solution. So set $\lambda_{1}^{\ast}=0$. Then we have%
\[
\left.  \frac{\partial L\left[  x,\lambda_{1},\lambda_{2}\right]  }{\partial
x}\right|  _{x=\frac{1}{2},\lambda_{1}=\lambda_{1}^{\ast},\lambda_{2}%
=\lambda_{2}^{\ast}}=\left.  -\left(  2x-2\right) -\lambda_{1} +\lambda_{2}\right|
_{x=\frac{1}{2},\lambda_{1}=\lambda_{1}^{\ast},\lambda_{2}=\lambda_{2}^{\ast}%
}=1-1=0
\]
just as we want. You can check that the other two inequalities hold with
complementary slackness as the theorem suggests.

So the idea is to find the optimal solution, then calculate how much could be
gained by violating the constraints so that you can set appropriate penalties
to prevent that. You may also see that if we pick the penalties properly, you
won't want to violate the constraints, so penalties will never actually be
paid and the value of the Lagrangian at the optimal solution using the
$\lambda^{\ast}$'s will be equal to the value of the original objective
evaluated at the optimum.

You might notice that this requires that we actually know the solution to the
problem in order to be able to calculate the appropriate penalties. This seems
to defeat the purpose of using this to find a solution in the first place. You
don't actually use this to method to calculate solutions (though it will
sometimes take you that far). You use it to describe the properties of solutions.

An obvious problem to try this with is the problem of maximizing a utility
function subject to a budget constraint. In particular, suppose that the
utility function is $u\left(  x,y\right)  =x^{\alpha}y^{\left(  1-\alpha
\right)  }$ where $0<\alpha<1$ is some constant. We need to satisfy the budget
constraint $M\geq px+qy$ where $p$ and $q$ are the prices for $x$ and $y$
respectively. We also need both consumption levels to be non-negative. So to
put this in the form needed by the theorem let $c_{1}=-M$ and $G_{1}\left(
x,y\right)  =-px-qy$, $c_{2}=c_{3}=0$, with $G_{2}\left(  x,y\right)  =x$ and
$G_{3}\left(  x,y\right)  =y$. 

The Lagrangian is then%
\[
x^{\alpha}y^{\left(  1-\alpha\right)  }+\lambda_{1}\left[  -M-\left(
-px-qy\right)  \right]  +\lambda_{2}[-x]+\lambda_{3}\left[  -y\right]
\]
I don't know what the optimal solution to this problem is. Even if I find a
solution to the first order conditions described above, this doesn't really
help since the theorem \emph{does not} say that things that satisfy the first
order conditions are solutions, it is the other way around.

Nonetheless I can combine things that I do know about the solution to make
some progress. For example if there is positive income $M>0$, then I should
never pick $x=0$ or $y=0$ since this will make the objective $0$ and I can get
strictly more than that if I make both $x$ and $y$ strictly positive, even if
they are very small. Furthermore, if I increase $x$ or $y$ or both of them a
little bit, I strictly increase the object. So the solution can never lie
strictly inside the budget constraint. So I already know that I will need to
impose a penalty to prevent myself from exceeding the budget, though I won't
want to impose any penalties for the non-negativity constraints since I would
never want to violate them anyway.

So at the optimal solution I will have constants $\lambda_{1}$, $\lambda_{2}$
and $\lambda_{3}$ such that%
\[
\frac{\partial L\left[  x,y,\lambda_{1},\lambda_{2},\lambda_{3}\right]
}{\partial x}=\alpha x^{\alpha-1}y^{\left(  1-\alpha\right)  }+\lambda
_{1}p-\lambda_{2}=0
\]%
\[
\frac{\partial L\left[  x,y,\lambda_{1},\lambda_{2},\lambda_{3}\right]
}{\partial y}=\left(  1-a\right)  x^{\alpha}y^{-\alpha}+\lambda_{1}%
q-\lambda_{3}=0
\]%
\[
-M-\left(  -px-qy\right)  \leq0;\lambda_{1}\leq0
\]%
\[
-x\leq0;\lambda_{2}\leq0
\]%
\[
-y\leq0;\lambda_{3}\leq0
\]
where the last three conditions hold with complimentary slackness. Since I
know that $x$ and $y$ both have to be strictly positive at the optimal
solution, then I also know that $\lambda_{2}=\lambda_{3}=0$ by complementary
slackness. Also $M$ should exactly equal $px+qy$ and the penalty for violating
the constraint $\lambda_{1}$ should be strictly negative. So this leaves me
with%
\[
\alpha x^{\alpha-1}y^{\left(  1-\alpha\right)  }+\lambda_{1}p=0
\]%
\[
\left(  1-a\right)  x^{\alpha}y^{-\alpha}+\lambda_{1}q=0
\]%
\[
M=px+qy
\]
These three equations in the three unknowns must have a solution from the
theorem above (at least provided you are sure that a solution exists). Divide
the first equation by the second to get%
\[
\frac{\alpha}{\left(  1-\alpha\right)  }\frac{y}{x}=\frac{p}{q}%
\]
or $y=\frac{px}{q}\frac{\left(  1-\alpha\right)  }{\alpha}$. Substitute this
into the final equation in the first order condition to get%
\[
M=px+q\frac{px}{q}\frac{\left(  1-\alpha\right)  }{\alpha}=px\left[
\frac{\alpha+\left(  1-\alpha\right)  }{\alpha}\right]
\]
or $x=\frac{\alpha M}{p}$. Similarly, $y=\frac{(1-\alpha) M}{q}$. So, you can give a
complete characterization of the result by using the first order conditions, along with the properties that you know about solution. 
\end{document}