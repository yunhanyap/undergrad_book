%% This document created by Scientific Word (R) Version 3.0

\documentclass[12pt]{article}
\usepackage{graphicx}
\usepackage{amsmath}
\usepackage{amsfonts}
\usepackage{amssymb}
%TCIDATA{OutputFilter=latex2.dll}
%TCIDATA{CSTFile=LaTeX article (bright).cst}
%TCIDATA{Created=Sun Oct 31 13:36:06 2004}
%TCIDATA{LastRevised=Wed Nov 03 10:10:02 2004}
%TCIDATA{<META NAME="GraphicsSave" CONTENT="32">}
%TCIDATA{<META NAME="DocumentShell" CONTENT="General\Blank Document">}
%TCIDATA{Language=American English}
\newtheorem{theorem}{Theorem}
\newtheorem{acknowledgement}[theorem]{Acknowledgement}
\newtheorem{algorithm}[theorem]{Algorithm}
\newtheorem{axiom}[theorem]{Axiom}
\newtheorem{case}[theorem]{Case}
\newtheorem{claim}[theorem]{Claim}
\newtheorem{conclusion}[theorem]{Conclusion}
\newtheorem{condition}[theorem]{Condition}
\newtheorem{conjecture}[theorem]{Conjecture}
\newtheorem{corollary}[theorem]{Corollary}
\newtheorem{criterion}[theorem]{Criterion}
\newtheorem{definition}[theorem]{Definition}
\newtheorem{example}[theorem]{Example}
\newtheorem{exercise}[theorem]{Exercise}
\newtheorem{lemma}[theorem]{Lemma}
\newtheorem{notation}[theorem]{Notation}
\newtheorem{problem}[theorem]{Problem}
\newtheorem{proposition}[theorem]{Proposition}
\newtheorem{remark}[theorem]{Remark}
\newtheorem{solution}[theorem]{Solution}
\newtheorem{summary}[theorem]{Summary}
\newenvironment{proof}[1][Proof]{\textbf{#1.} }{\ \rule{0.5em}{0.5em}}

\begin{document}
\title{Expected Utility and Risk Aversion}
\author{Michael Peters}
\date{\today}
\maketitle

\section{Introduction}

This reading describes how peoples' aversion to risk affects decisions they
make about investment. Basically, the concepts used to do this analysis emerge
naturally when people have expected utility preferences, but not otherwise. So,
it illustrates one important way that expected utility is applied.

This analysis also makes it possible to illustrate how to do \emph{comparative
statics}. Comparative statics typically involve calculations designed to show
the direction in which changes in the environment move people's optimal
decisions. Convincing comparative static results are ones that hold even if
you only impose weak restrictions on preferences. So, for instance, to explain
how an increase in price affects a buyer's demand when he or she has
Cobb-Douglas preferences is not very convincing because Cobb Douglas
preferences are a very special case. In fact, as we have already shown, an
(uncompensated) increase in price will only reduce demand under very special
conditions. That is why the method for doing comparative statics tend to be a
little more sophisticated, and the questions asked tend not to be the most
obvious ones.

\subsection{Risk Aversion}

The exercise we want to do involves measuring how much people dislike risk.
One good way to do this simply asks how much a consumer would be willing to
pay to avoid a \emph{fair} bet. A\ fair bet is a lottery whose expected payoff
is equal to zero. Such lotteries are interesting because you can't really gain
or lose by playing them in the long run. So, they aren't intrinsically
valuable. On the other hand, when the expected payoff is zero, that means you
could either gain or lose money by playing them. They expose you to risk, and
it is the aversion to risk we are trying to measure.

Let's assume that all the lotteries we are interested in involve monetary
consequences. Further, let's refer to a lottery by naming the \emph{probability
distribution function} associated with it.\ The probability distribution
function of a lottery $F$ gives for each real monetary payoff $x\in\mathbb{R}%
$, the probability that the monetary consequence associated with the lottery
will be less than or equal to $x$. In this case, the monetary consequence is
just a random variable. The expected payoff associated with this random
variable $x$ is%
\[
\mathbb{E}x\mathbb{=}\int xF^{\prime}\left(  x\right)  dx
\]
assuming that the probability distribution function has a derivative (called
the probability \emph{density} function).

Let $W$ be any level of wealth. The \emph{risk premium} that our consumer
would be willing to pay to avoid the lottery $F$ when his wealth is $W$ is
just the solution to the equation%
\[
U\left(  W-p\right)  =U\left(  W,F\right)
\]

WHY IS U CAPITALIZED HERE AND NOT ANYWHERE ELSE?

where $U$ is the consumer's utility function for lotteries $F$. The argument
$W-p$ on the left hand side, just means the lottery where the consumer
receives $W-p$ for sure. It makes sense that the more a consumer is willing to
pay to avoid this lottery, the more risk averse he or she must be. That isn't
very helpful, though, because the consumers risk aversion measured this way is
going to depend on $W$ and $F$ in fairly complicated ways. Once we get to
applications, this simply won't help very much (except in an abstract sense).

Now we can try to make the idea more precise using expected utility. The
independence axiom (among other things) says that the equality above can be
written as%
\[
u\left(  W-p\right)  =\int u\left(  w+x\right)  F^{\prime}\left(  x\right)
dx
\]
Notice one of the advantages of the expected utility theorem. The function $u$
(or the utility for wealth function) doesn't depend on the lottery or
lotteries we are tying to evaluate, so we can use exactly the same function on
both sides of the equation. This has made it possible to disentangle the
interaction between $F$ and $W$ when evaluating the relationship. Furthermore,
the function $U$ is a function whose argument is a function - hard to work
with. The independence axiom gives us functions whose arguments are real
numbers. Now some of the tools we have learned can be brought to bear.

First of all, suppose we write out the first order Taylor approximation of the
function on the left hand side of the equation around $W$. This will give us%
\[
u\left(  W\right)  -u^{\prime}\left(  W\right)  p
\]
This is just an approximation to the exact value. If the utility for wealth
function is concave, then this approximation will be too large, but the
approximation will be pretty good if $p$ is small.

We can't quite do the same thing on the right hand side, because some of the
realizations of $x$ will be pretty big even if they are very unlikely. We can
correct for this somewhat by using the second order version of the Taylor
expansion on the right hand side, to get%
\[
\int\left\{  u\left(  W\right)  +u^{\prime}\left(  W\right)  x+u^{\prime
\prime}\left(  w\right)  \frac{x^{2}}{2}\right\}  F^{\prime}\left(  x\right)
dx
\]
This will be a pretty good approximation to the right hand side provided that
it isn't very likely that the realization of $x$ will be large (in other
words, most of the probability in F will be assigned to monetary consequences
that are pretty close to zero). If we just blindly treat these two
approximations as if they were exact reproductions, then we would get the
following interesting equation (canceling the common terms on both sides and
using the fact that $\int xF^{\prime}\left(  x\right)  dx=0$)%
\[
-u^{\prime}\left(  W\right)  p=\int u^{\prime\prime}\left(  W\right)
\frac{x^{2}}{2}F^{\prime}\left(  x\right)  dx
\]
or
\[
p=-\frac{u^{\prime\prime}\left(  W\right)  }{u^{\prime}\left(  W\right)  }%
\int\frac{x^{2}}{2}F^{\prime}\left(  x\right)  dx
\]
This equation says that the risk premium is made up of two multiplicatively
separable parts. One part is the integral, which depends only on $F$. The
other part is the ratio of the second to first derivatives of $u$. In words,
this says that the risk premium will go up as the lottery $F$ changes (and
specifically it will go up as $\int\frac{x^{2}}{2}F^{\prime}\left(  x\right)
dx$ changes) and as the utility function $u$ changes. The first term is then a
legitimate measure of the disutility of risk that comes purely from the
consumers own preferences.

The term%
\[
-\frac{u^{\prime\prime}\left(  W\right)  }{u^{\prime}\left(  W\right)  }%
\]
is referred to as the Arrow Pratt measure of absolute risk aversion.

We will shortly see how this measure can be used to get some insight into the
way that investment decisions work. For the moment, the main thing to note is
that the Arrow Pratt measure is a conceptual device that emerges with the help
of the expected utility theorem. The expected utility theorem tells you first
of all that it makes sense to think about something called a utility for
wealth function. Of course, nothing in this story says that wealth actually
conveys utils or gives direct pleasure to the person who has it. The gist of
the story is that consumers whose choices obey the independence axiom will act
as if wealth gives them pleasure.

Now the story goes a bit further. When they act this way, they also seem to
assign a monetary value to certain types of risk that depends only on their
status quo wealth, and something that looks a little bit like the variance of
the risk that they face. One natural assumption would seem to be that the risk
premium that a consumer would be willing to pay to avoid risk would be lower
if the consumer were more wealthy. The formulation so far shows exactly how to
formalize this idea - the function $-\frac{u^{\prime\prime}\left(  w\right)
}{u^{\prime}\left(  w\right)  }$ should be a decreasing function. We will come
back to this idea momentarily.

\subsection{The Portfolio Problem}

A fairly simple version of the portfolio problem can be studied by assuming
that there are exactly two different securities that an investor can buyer. A
\emph{security} is a lottery. The consequences of the lottery are possible
rates of return on investment. If the consequence of the lottery is a rate of
return $s$, then the security pays $1+s$ dollars tomorrow for each dollar that
is invested in the lottery today. In the problem that we are going to analyze,
one of the two securities is safe in the sense that the lottery that produces
the rate of return gives a rate of return of zero for sure. So, each dollar
invested today gives back exactly one dollar tomorrow, no more and no less.
There is also a risky security where the probability distribution function for
the random rate of return is $F$.\ Sometimes this rate of return will be
positive and the security will pay back more than one dollar for each dollar
of investment. Other times the rate of return will actually be negative, and a
dollar invested will return less than one dollar tomorrow.

The investor has $w$ WHY ISN'T $w$ CAPITALIZED NOW AND BELOW?  dollars to invest in these two securities. His \emph{ex
post} income (i.e., his income tomorrow after the rate of return on the
lottery is realized) when he invests $i_{s}$ in the safe security and $i_{r}$
in the risky security, is given by%
\[
i_{s}+i_{r}\left(  1+s\right)
\]

A pair $\left(  i_{s},i_{r}\right)  $ is called a \emph{portfolio}. Each
portfolio generates a different lottery over monetary outcomes. Then relying
on the independence axiom, we can invoke the expected utility theorem and
conclude that there is a utility for wealth function $u$ such that one
portfolio (and its associated lottery) $\left(  i_{s},i_{r}\right)  $ is
preferred to another $\left(  i_{s}^{\prime},i_{r}^{\prime}\right)  $ if%
\[
\int u\left(  i_{s}+i_{r}\left(  1+s\right)  \right)  F^{\prime}\left(
s\right)  ds\geq
\]%
\[
\int u\left(  i_{s}^{\prime}+i_{r}^{\prime}\left(  1+s\right)  \right)
F^{\prime}\left(  s\right)  ds.
\]

If that is true, then the investor should choose the portfolio that maximizes%
\[
\int u\left(  i_{s}+i_{r}\left(  1+s\right)  \right)  F^{\prime}\left(
s\right)  ds
\]
subject to the constraints that%
\[
i_{s}+i_{r}\leq W
\]%
\[
i_{r}\geq0
\]%
\[
i_{s}\geq0
\]

We could solve this using Lagrangian methods, recalling that before you do so,
you need to tease out some properties of the solution in order to guess which
constraints are likely to be important. To see a way to do this, simply
imagine that the integral above is just a particular function%
\[
U\left(  i_{s},i_{r}\right)  \equiv
\]%
\[
\int u\left(  i_{s}+i_{r}\left(  1+s\right)  \right)  F^{\prime}\left(
s\right)  ds
\]
and that the first constraint above is just a budget constraint where the
prices of both securities are just equal to $1$. So, we should be able to find
the optimal portfolio by finding the highest indifference curve that touches
the budget constraint, as the indifference curve $II$ does in Figure $1$.%
%TCIMACRO{\FRAME{ftbpF}{2.0263in}{1.6302in}{0pt}{}{}{risk_aversion_fig1.eps}%
%{\special{ language "Scientific Word";  type "GRAPHIC";
%maintain-aspect-ratio TRUE;  display "USEDEF";  valid_file "F";
%width 2.0263in;  height 1.6302in;  depth 0pt;  original-width 1.9882in;
%original-height 1.5939in;  cropleft "0";  croptop "1";  cropright "1";
%cropbottom "0";  filename 'risk_aversion_fig1.eps';file-properties "XNPEU";}%
%}}%
%BeginExpansion
\begin{figure}
[ptb]
\begin{center}
\includegraphics[
height=1.6302in,
width=2.0263in
]%
{risk_aversion_fig1.eps}%
\end{center}
\end{figure}
%EndExpansion

The slope of the indifference curve is given by the marginal utility of the
good on the horizontal axis (i.e., $i_{s}$) divided by the marginal utility
associated with the good on the vertical axis (i.e., $i_{r}$). Differentiating
gives%
\[
-\frac{\frac{\partial U\left(  i_{s},i_{r}\right)  }{\partial i_{s}}}%
{\frac{\partial U\left(  i_{s},i_{r}\right)  }{\partial i_{r}}}=
\]%
\[
-\frac{\int u^{\prime}\left(  i_{s}+i_{r}\left(  1+s\right)  \right)
F^{\prime}\left(  s\right)  ds}{\int u^{\prime}\left(  i_{s}+i_{r}\left(
1+s\right)  \right)  \left(  1+s\right)  F^{\prime}\left(  s\right)  ds}%
\]

Now following our usual procedure, we can try to check the conditions under
which we might find a solution at the corner where all of wealth is invested
in the riskless asset. This would occur if the indifference curve happened to
be steeper than the budget line. So, let's evaluate the slope of the
indifference curve at the bundle $\left(  w,0\right)  $. Substituting into the
formula above gives%
\[
-\frac{\int u^{\prime}\left(  w\right)  F^{\prime}\left(  s\right)  ds}{\int
u^{\prime}\left(  w\right)  \left(  1+s\right)  F^{\prime}\left(  s\right)
ds}=
\]%
\[
-\frac{1}{1+\int sF^{\prime}\left(  s\right)  ds}%
\]
This means that whether or not the investor optimally invests in the risky
asset depends entirely on $\int sF^{\prime}\left(  s\right)  ds$. If this is
positive, the (absolute value of the) slope of the indifference curve is
smaller than $1$ and the tangency has to occur somewhere up the budget line to
the left of $\left(  w,0\right)  $. On the other hand, if the mean value of
$s$ is less than zero, then the indifference curve will be steeper than the
budget line, and the optimal solution will be at the corner of the budget set.

This is called the \emph{diversification theorem}. If there is a risky asset
whose expected return exceeds the expected return on the safe asset, then the
optimal portfolio will always involve some investment in the risky asset.

\subsubsection{Comparative Statics and Wealth}

One assertion that seems that it must be true is that wealthy people invest
more in risky assets. Surprisingly, this is not always the case. It is
difficult to see why this might be by using intuition alone. As I have often
mentioned, this is when mathematics can be very useful. Intuition is rarely
wrong, is just doesn't give the whole story. The math can often help you think
out the parts that you tend to gloss over when you think intuitively. Often
the biggest insights in economics come by understanding the complexities that
intuition can't see.

The problem that is discussed in this section also provides an opportunity to
see the way that comparative statics is often done in applications. We are
interested in what the effect of a change in wealth $w$ will be on the optimal
portfolio. One way to figure this out is to try to find out how a change in
wealth will affect the position of the tangency between the indifference curve
and budget line. To see how this line of argument works, start by substituting
the fact that $i_{s}=w-i_{r}$ into the tangency condition to get%
\[
\frac{\int u^{\prime}\left(  w-i_{r}+i_{r}\left(  1+s\right)  \right)
F^{\prime}\left(  s\right)  ds}{\int u^{\prime}\left(  w-i_{r}+i_{r}\left(
1+s\right)  \right)  \left(  1+s\right)  F^{\prime}\left(  s\right)  ds}=1
\]
Simplifying the arguments of the functions gives%
\[
\frac{\int u^{\prime}\left(  w+i_{r}s\right)  F^{\prime}\left(  s\right)
ds}{\int u^{\prime}\left(  w+i_{r}s\right)  \left(  1+s\right)  F^{\prime
}\left(  s\right)  ds}=1
\]
Multiplying both sides by the denominator on the left gives%
\[
\int u^{\prime}\left(  w+i_{r}s\right)  F^{\prime}\left(  s\right)  ds-\int
u^{\prime}\left(  w+i_{r}s\right)  \left(  1+s\right)  F^{\prime}\left(
s\right)  ds=0
\]
Canceling the common terms gives the equation%
\begin{equation}
\int u^{\prime}\left(  w+i_{r}s\right)  sF^{\prime}\left(  s\right)
ds=0\label{FOC}%
\end{equation}

The trick at this point is to assume that $i_{s}$ is actually a function of
$w$ that adjusts in such a way that the equation above is always satisfied.
Then, in fact,%
\[
\int u^{\prime}\left(  w+i_{r}\left[  w\right]  s\right)  sF^{\prime}\left(
s\right)  ds\equiv0
\]
Since this holds uniformly under this definition, we can differentiate both
sides of the expression with respect to $w$ and the derivatives will also be
equal. In other words%
\[
\int u^{\prime\prime}\left(  w+i_{r}\left[  w\right]  s\right)  sF^{\prime
}\left(  s\right)  ds+\int u^{\prime\prime}\left(  w+i_{r}\left[  w\right]
s\right)  \frac{di_{r}\left[  w\right]  }{dw}s^{2}F^{\prime}\left(  s\right)
ds=0
\]
Solving for the derivative gives%
\[
\frac{di_{r}\left[  w\right]  }{dw}=-\frac{\int u^{\prime\prime}\left(
w+i_{r}\left[  w\right]  s\right)  sF^{\prime}\left(  s\right)  ds}{\int
u^{\prime\prime}\left(  w+i_{r}\left[  w\right]  s\right)  s^{2}F^{\prime
}\left(  s\right)  ds}%
\]
We want to know how an increase in $w$ will change the amount invested in the
risky asset. In other words, we want to know whether the derivative of the
optimal value of $i_{r}$ with respect to a change in wealth will be positive.
This expression almost gives the answer. The denominator is an integral. Each
term in the integrand in negative provided the investor is risk averse (which
gives $u^{\prime\prime}<0$). There is a minus sign in front of the fraction,
and minus times minus is positive. So, we could conclude that the derivative is
positive if we could show that the numerator is positive. Unfortunately, this
is not obvious if it is true. The derivative $u^{\prime\prime}$ is certainly
negative, but $s$ can be either positive or negative. The sign of the integral
will depend on how big $u^{\prime\prime}$ is when $s$ is negative compared
with how big it is when $s$ is positive.

The lesson of the comparative statics has now been discovered. To conclude
that increases in wealth raise investment, we need more information. Or, we
need to restrict the set of preferences that we think are plausible.
Fortunately, there is a fairly easy restriction that will do this trick. We
have described the Arrow Pratt measure of absolute risk aversion
$\frac{u^{\prime\prime}\left(  w\right)  }{u^{\prime}\left(  w\right)  }$. It
is proportional to the size of the risk premium associated with fair gambles.
Suppose we assumed that this measure of risk aversion is decreasing with
wealth. That would mean that we would be assuming (or restricting attention
to) investors whose risk premium falls as they become more wealthy. Surely
these investors must raise investment as their wealth increases. So, let's check
this out.

Write out the Arrow Pratt measure as it appears in our comparative static
equation. Then we would have%
\[
-\frac{u^{\prime\prime}\left(  w+i_{r}s\right)  }{u^{\prime}\left(
w+i_{r}s\right)  }\leq-\frac{u^{\prime\prime}\left(  w\right)  }{u^{\prime
}\left(  w\right)  }%
\]
whenever $s>0$ by the assumption that the Arrow Pratt measure is decreasing.
Well the whole problem arises from the fact that we don't know that $s$ is
positive. So, let's try a trick. If $s$ is positive, it must also be true that%
\[
-\frac{u^{\prime\prime}\left(  w+i_{r}s\right)  }{u^{\prime}\left(
w+i_{r}s\right)  }s\leq-\frac{u^{\prime\prime}\left(  w\right)  }{u^{\prime
}\left(  w\right)  }s
\]
The nice thing about this expression is that if we change the sign of $s$ to
negative (which of course means that $-\frac{u^{\prime\prime}\left(
w+i_{r}s\right)  }{u^{\prime}\left(  w+i_{r}s\right)  }\geq-\frac
{u^{\prime\prime}\left(  w\right)  }{u^{\prime}\left(  w\right)  }$) then it
would still be true that%
\[
-\frac{u^{\prime\prime}\left(  w+i_{r}s\right)  }{u^{\prime}\left(
w+i_{r}s\right)  }s\leq-\frac{u^{\prime\prime}\left(  w\right)  }{u^{\prime
}\left(  w\right)  }s
\]
So, this last expression is actually correct no matter what the sign of $s$. So,
let's multiply both sides of this inequality by $u^{\prime}\left(
w+i_{r}s\right)  $ then integrate the result over $s$ to get%
\[
-\int u^{\prime\prime}\left(  w+i_{r}s\right)  sF^{\prime}\left(  s\right)
ds\leq-\frac{u^{\prime\prime}\left(  w\right)  }{u^{\prime}\left(  w\right)
}\int u^{\prime}\left(  w+i_{r}s\right)  sF^{\prime}\left(  s\right)  ds
\]
If you look back, you will see that the right hand side of this equation is
proportional to the left hand side of (\ref{FOC}) which is zero. So, we have%
\[
-\int u^{\prime\prime}\left(  w+i_{r}s\right)  sF^{\prime}\left(  s\right)
ds\leq0
\]
which is just the result we wanted (since it shows that $\int u^{\prime\prime
}\left(  w+i_{r}s\right)  sF^{\prime}\left(  s\right)  ds\geq0$). 

After all this work, what we have discovered is that an investor will increase
his investment in the risky asset as his wealth rises provided his Arrow Pratt
measure of absolute risk aversion is decreasing as his wealth increases. 
\end{document}