%% LyX 2.1.4 created this file.  For more info, see http://www.lyx.org/.
%% Do not edit unless you really know what you are doing.
\documentclass{article}
\usepackage[T1]{fontenc}
\usepackage[latin9]{inputenc}
\usepackage{esint}
\begin{document}

\title{Auctions}

\maketitle
In an auction, a single seller tries to sell one unit of some commodity
to one of $n$ different buyers. Only one buyer can buy the unit.
The seller's problem is to decide who to sell it to, and how much
to charge them. We'll imagine that sellers are risk neutral expected
revenue maximizers. They just want to sell the good in a way that
will maximize their expected revenue. Buyers are interested in the
difference between what an object is worth to them, and what they
have to pay for it. They want to maximize the product of the probability
that they win the auction times the difference between their value
and what they expect to pay for the good when they win.

The thing that makes everyone's problem hard is that no one knows
any of the buyers' values. It is a game of incomplete information.
In everything that follows, we'll make the assumption that values
are identically and independently distributed. To keep things simple,
lets just suppose that this distribution has its support on the interval
$\left[0,1\right]$, meaning that if $F\left(x\right)$ is the probability
that a certain bidder's valuation is less than or equal to $x$, then
$F\left(0\right)=0$ and $F\left(1\right)=1$. Otherwise, lets suppose
this distribution has a density given by $f\left(x\right)$. This
means, of course, that $F\left(x\right)=\int_{0}^{x}f\left(t\right)dt$. 

The idea here is that each bidder believes that each of the other
bidders has a value that is somewhere between $0$ and $1$, and that
$F\left(x\right)$ is the probability that this value is less than
or equal to $x$. 


\section*{Second price auction}

In a sealed bid second price auction, each bidder submits a bid to
the seller. The seller then chooses the bidder who submits the highest
bid, and offers him the good at a price which is equal to the second
highest bid that was submitted.

As we argued, the bidder can't affect the price he pays by changing
his bid. The price he pays is the bid that someone else submitted.
However, he can affect the probability that he wins by raising or
lowering his bid. If his bid is lower than his value, he will raise
the probability of winning the auction by increasing his bid. This
would be good, because he will be better off whenever he wins the
auction while paying something less than his value. On the other hand,
if his bid is above his value, then he is inevitably doing to win
sometimes and pay more for the good than it is actually worth to him.
For that reason, he would be better off lowering his bid. We decided
that the bid $b\left(v\right)$ that maximizes the expected payoff
for a buyer whose value is $v$ is just $b\left(v\right)=v$. 

Lets check some of the implications of this. The first is just the
bidder who wins in this case is just the bidder with the highest value.
So if a bidder wants to figure out how likely it is he or she will
win in equilibrium, they just compute the probability that their value
is highest. This is the probability that each of the other bidders
has a lower value than they do, and this is just $F\left(v\right)^{n-1}$.
So the probability they win the auction when their value is $v$ is
going to be $F\left(v\right)^{n-1}$. 

What are they likely to pay if they win? This is the expectation of
the highest value of the other bidders, conditional on the other bidders
having values lower than $v$. This is
\begin{equation}
\frac{\left(n-1\right)\int_{0}^{v}\tilde{v}F^{n-2}\left(\tilde{v}\right)f\left(\tilde{v}\right)d\tilde{v}}{F^{n-1}\left(v\right)}.\label{expected-payment}
\end{equation}
Lets go over this calculation. Pick one of the other bidders, say
bidder 1, and imagine that his value is $\tilde{v}$. This event occurs
with probability $f\left(\tilde{v}\right)$. In that case, the probability
that all the others have lower values than his is $F\left(\tilde{v}\right)^{n-2}$,
because there are $n-2$ bidders other than you and bidder 1. Of course,
it could also have been bidder 2 who had this value $\tilde{v}$.
Summing this up over the $n-1$ bidders other than yourself, this
says that the probability that the highest value bidder among the
others has value $\tilde{v}$ is $\left(n-1\right)F\left(\tilde{v}\right)^{n-2}f\left(\tilde{v}\right)$.

Now we want to use Bayes rule. You know that your value is $v$. So
the joint probability that your value is $v$ and the highest among
the others is $\tilde{v}$ is just $\left(n-1\right)F\left(\tilde{v}\right)^{n-2}f\left(\tilde{v}\right)$,
as we just calculated as long as $\tilde{v}<v$. For Bayes rule, we
then need to divide by the probability that your value $v$ is highest,
which is just $F^{n-1}\left(v\right)$. We then take the expectation
using this conditional probability distribution.

In words, we just decided that the amount you should expect to pay
if your value is $v$ in a second price auction is given by (\ref{expected-payment}).


\subsection*{Problems}
\begin{enumerate}
\item Work out the expected payment when there are 2 other bidders and $F$
is uniform (i.e. $F\left(x\right)=x$). Now do the same when there
are three other bidders. How does the amount you expect to pay change
between 2 and 3 other bidders?
\item Answer question 1 again, but assume that $F\left(x\right)=x^{2}$.
What impact does this change in the distribution have.
\end{enumerate}

\subsection*{Why worry about the expected payment?}

If you bid in a second price auction, you will do okay as long as
you don't bid more than your value. In a way, there isn't really much
reason to do the calculation we did above. However, it is an important
calculation for the seller. Lets do the calculation from the seller's
point of view. Suppose that $v$ is the highest value among the bidders
in the auction. Then the revenue that the seller should expect to
get from the bidder who wins the auction is exactly what that bidder
expects to pay, i.e., the expression (\ref{expected-payment}). The
probability that bidder 1 is the high bidder and has value $v$ is
\[
f\left(v\right)F^{n-1}\left(v\right)
\]


Multiply that by the revenue the seller expects to earn from bidder
1 when he is high bidder with value $v$ and you get
\[
\left(n-1\right)\int_{0}^{v}\tilde{v}F^{n-2}\left(\tilde{v}\right)f\left(\tilde{v}\right)d\tilde{v}f\left(v\right).
\]
Now integrate this over all the possible values bidder 1 could have,
then multiply it by $n$ because there are $n$ bidders in all, and
you get the revenue the seller expects to get from the second price
auction
\begin{equation}
n\int_{0}^{1}\left(n-1\right)\left\{ \int_{0}^{v}\tilde{v}F^{n-2}\left(\tilde{v}\right)f\left(\tilde{v}\right)d\tilde{v}\right\} f\left(v\right)dv.\label{second-price-revenue}
\end{equation}



\subsection*{Problems}
\begin{enumerate}
\item Calculate expected revenue when $F\left(v\right)=v$ and show that
it is equal to $\frac{\left(n-1\right)}{n}.$
\end{enumerate}

\subsection*{First Price Auctions.}

It is interesting that second price auctions have an equilibrium where
bidders bid their true values. Yet one might wonder whether there
might not be better ways to sell something. For example, imagine that
you are trying to sell some public land to make money for taxpayers.
You decide to hold a second price auction. Some big company gives
you a bid of \$1 million. By what we have just said, that is the amount
the company thinks the land is worth. Why not just charge them \$1
million - that seems better for taxpayers. After all, why deliberately
charge the company something less than what you know they are willing
to pay.

The answer is that if they know you are going to do this, they won't
bid \$1 million, they will bid something considerably less. If you
want to figure out if it would be better to charge them what they
bid, you need to figure out exactly what they will bid.

To do this we can use an unusual conceptual approach. Suppose that
we guess that the companies will use a common bidding rule $b\left(v\right)$.
What they bid will depend on their value, which we don't know, but
if two firms have the same value, then we expect they will submit
the same bid. Lets also assume that whatever this bidding rule is,
it is monotonically increasing. Suppose there are $n$ companies bidding
on the land and take the perspective of any one of them. 

Since it is a first price auction, it is much simpler to figure out
what the company will pay if it wins - just whatever it chose to bid.
So we really only have to figure out how to find the probability they
win with each bid. Suppose the company decides to bid $b^{\prime}$.
Then if it thinks the other firms are using the bidding rule $b\left(v\right)$,
it expects to beat any of the other firms who bid less than $b^{\prime}.$
Since the bidding rule is monotonic, they should bid less than $b^{\prime}$
if $b\left(\tilde{v}\right)<b^{\prime}$. So what we need to do is
to figure out what value a firm would need to have to make them bid
$b^{\prime}.$ Since $b\left(v\right)$ is strictly increasing, this
value is $b^{-1}\left(b^{\prime}\right)$.

But that means that the bid $b^{\prime}$ will win the auction if
all the other bidders have values less than or equal to $b^{-1}\left(b^{\prime}\right)$.
This probability is given by
\[
F^{n-1}\left(b^{-1}\left(b^{\prime}\right)\right).
\]
So a bidder with value $v$ needs to maximize
\[
\left(v-b^{\prime}\right)F^{n-1}\left(b^{-1}\left(b^{\prime}\right)\right).
\]
Now we can use a trick. If the bidding function $b\left(v\right)$
is an equilibrium, then $b\left(v\right)$ will maximize the function
above for a bidder with value $v$. One way to say this is that bidder
$v$ would rather submit the bid $b\left(v\right)$ than the bid that
would be submitted by a bidder with some other value, say $v^{\prime}$.
That is, in a Bayes Nash equilibrium, bidder $v$ should prefer to
bid $b\left(v\right)$ to $b\left(v^{\prime}\right)$.

We don't know exactly what a bidder with value $v^{\prime}$ will
bid, but whatever it is, it will win if all the other bidders have
values less than $v^{\prime}$ because of the fact the bidding rule
is monotonic. So the bidding rule should satisfy
\[
\left(v-b\left(v\right)\right)F^{n-1}\left(v\right)\ge\left(v-b\left(v^{\prime}\right)\right)F^{n-1}\left(v^{\prime}\right)
\]
for all $v^{\prime}$. In particular, that means that the derivative
of the function 
\[
\left(v-b\left(v^{\prime}\right)\right)F^{n-1}\left(v^{\prime}\right)
\]
with respect to $v^{\prime}$should be zero when $v^{\prime}=v$.
In other words
\begin{equation}
\left(v-b\left(v\right)\right)\left(n-1\right)F^{n-2}\left(v\right)f\left(v\right)=b^{\prime}\left(v\right)F^{n-1}\left(v\right).\label{foc}
\end{equation}


One way we could approach this is to solve for 
\[
b^{\prime}\left(v\right)=\frac{\left(v-b\left(v\right)\right)\left(n-1\right)f\left(v\right)}{F\left(v\right)}.
\]
If you observe that must hold for every value of $v$, it becomes
a differential equation that we could try to solve. However, there
is another way to get the solution that will help in our comparison
to the second price auction. Lets just rewrite (\ref{foc}) as
\[
v\left(n-1\right)F^{n-2}\left(v\right)f\left(v\right)=b\left(v\right)\left(n-1\right)F^{n-2}\left(v\right)f\left(v\right)+b^{\prime}\left(v\right)F^{n-1}\left(v\right).
\]
Now observe that the right hand side of this expression is just the
derivative of $b\left(v\right)F^{n-1}\left(v\right)$ with respect
to $v$.

What that means is that uniformly in $b$
\[
\frac{d\left\{ b\left(v\right)F^{n-1}\left(v\right)\right\} }{dv}=v\left(n-1\right)F^{n-2}\left(v\right)f\left(v\right).
\]
Then we just use the fundamental theorem of calculus, and integrate
the derivative to get the function itself, i.e.
\[
b\left(v\right)F^{n-1}\left(v\right)=\int_{0}^{v}\tilde{v}\left(n-1\right)F^{n-2}\left(\tilde{v}\right)f\left(\tilde{v}\right)d\tilde{v},
\]
or
\[
b\left(v\right)=\frac{\int_{0}^{v}\tilde{v}\left(n-1\right)F^{n-2}\left(\tilde{v}\right)f\left(\tilde{v}\right)d\tilde{v}}{F^{n-1}\left(v\right)}.
\]


Now you can look back at the expression we got in (\ref{expected-payment})
describing the amount that a bidder in the second price auction expects
to pay conditional on winning - you will see it is exactly the same.
The stunning conclusion is that the amount that the seller should
expect to receive from the winning bidder is exactly the same in both
the first and second price auctions. They produce exactly the same
revenue. 


\subsection*{All Pay Auctions.}

If you don't find the relationship between the first and second price
auction surprising, here is an even more surprising result. Many auctions
(or at least things that act like auctions) have the property that
the high bidder wins the auction and pays whatever she bid. Yet everyone
else in the auction has to pay what they bid as well. If you think
that sounds unreasonable, that is in many ways what happens in education.
To get a job you spend a lot of money on education - the most educated
person gets the most desirable job. If you don't get the most desirable
job, you still have to pay for the education you received.

Many kinds of litigation are like this. One party sues the other,
then both lawyer up. The side that spends the most on lawyers wins
the case, but both sides have to pay their lawyers.\footnote{An interesting example of this kind of thing that pertains to another
part of this course is the companies that act as 'patent trolls'.
The way patent trolls work is to apply for, or buy very vague patents,
then suing a company for patent violation. Even if the patent doesn't
apply, the company who is being sued has to defend itself in court,
which requires them to lawyer up in the manner described above. The
patent troll then offers to settle out of court for an upfront payment,
which the company will normally pay. This is type of extortion which
is perfectly legal under US \emph{intellectual property }law. If you
are getting bored with auctions, here is a story about patent trolls
- https://www.youtube.com/watch?v=3bxcc3SM\_KA}

We can find an equilibrium for this sort of thing using the approach
above. Lets suppose the bidders use a monotonic rule $b\left(v\right)$
to decide how much to bid. Once again, if $b\left(v\right)$ is a
Bayesian equilibrium bidding rule, then the function
\[
vF^{n-1}\left(v^{\prime}\right)-b\left(v^{\prime}\right)
\]
should be maximized when $v^{\prime}=v$. 

The corresponding first order condition is
\[
v\left(n-1\right)F^{n-2}\left(v\right)f\left(v\right)=b^{\prime}\left(v\right).
\]
This is actually really easy because we can use the fundamental rule
of calculus right away to get
\[
b\left(v\right)=\int_{0}^{v}\tilde{v}\left(n-1\right)F^{n-2}\left(\tilde{v}\right)f\left(\tilde{v}\right)d\tilde{v}.
\]
If you compare this to the bid in the first price auction, it is much
smaller.

However, the total expected payments to the seller are
\[
n\int_{0}^{1}b\left(v\right)f\left(v\right)dv=
\]
\[
n\int_{0}^{1}\int_{0}^{v}\tilde{v}\left(n-1\right)F^{n-2}\left(\tilde{v}\right)f\left(\tilde{v}\right)d\tilde{v}f\left(v\right)dv.
\]
 If you compare this to our original formula for the revenue in the
second price auction, given by (\ref{second-price-revenue}), you
will see that it is exactly the same.
\end{document}
