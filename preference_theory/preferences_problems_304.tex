%% This document created by Scientific Word (R) Version 3.0

\documentclass[12pt]{article}
\usepackage{graphicx}
\usepackage{amsmath}
\usepackage{amsfonts}
\usepackage{amssymb}
%TCIDATA{OutputFilter=latex2.dll}
%TCIDATA{CSTFile=LaTeX article (bright).cst}
%TCIDATA{Created=Sun Oct 17 16:41:08 2004}
%TCIDATA{LastRevised=Sun Oct 17 16:45:20 2004}
%TCIDATA{<META NAME="GraphicsSave" CONTENT="32">}
%TCIDATA{<META NAME="DocumentShell" CONTENT="General\Blank Document">}
\newtheorem{theorem}{Theorem}
\newtheorem{acknowledgement}[theorem]{Acknowledgement}
\newtheorem{algorithm}[theorem]{Algorithm}
\newtheorem{axiom}[theorem]{Axiom}
\newtheorem{case}[theorem]{Case}
\newtheorem{claim}[theorem]{Claim}
\newtheorem{conclusion}[theorem]{Conclusion}
\newtheorem{condition}[theorem]{Condition}
\newtheorem{conjecture}[theorem]{Conjecture}
\newtheorem{corollary}[theorem]{Corollary}
\newtheorem{criterion}[theorem]{Criterion}
\newtheorem{definition}[theorem]{Definition}
\newtheorem{example}[theorem]{Example}
\newtheorem{exercise}[theorem]{Exercise}
\newtheorem{lemma}[theorem]{Lemma}
\newtheorem{notation}[theorem]{Notation}
\newtheorem{problem}[theorem]{Problem}
\newtheorem{proposition}[theorem]{Proposition}
\newtheorem{remark}[theorem]{Remark}
\newtheorem{solution}[theorem]{Solution}
\newtheorem{summary}[theorem]{Summary}
\newenvironment{proof}[1][Proof]{\textbf{#1.} }{\ \rule{0.5em}{0.5em}}

\begin{document}

\begin{center}
Problem Set
\end{center}

\begin{enumerate}
\item  Draw a graph showing the indifference curve associated with each of the
following utility functions when the level of utility is equal to 1

\begin{itemize}
\item $u\left(  x,y\right)  =\max\left\{  x,y\right\}  $

\item $u\left(  x,y\right)  =x+2y$

\item $u\left(  x,y\right)  =x-2y$
\end{itemize}

\item  Someone has an endowment of \$20 in period 1 and \$40 in period 2. Her
utility is $u\left(  c_{1},c_{2}\right)  =c_{1}+c_{2}$. She can put money in a
bank account in period 1 and get it back again in period 2 without interest.
She can borrow money from a loan shark. For each dollar she borrows in period
$1$ she must pay back $\$1.50$ in period $2$. She can also buy a chair in
period 1 for \$10 that she knows that she will be able to resell in period $2$
for $\$15$.

\begin{itemize}
\item  Draw her budget line assuming that she does not buy the chair
(labelling you diagram carefully).

\item  Draw her budget line if she does buy the chair.

\item  Should she buy the chair or not? What consumption bundle should she
pick in each case ?

\item  Predict her consumption and whether or not she will buy the chair if
her utility function is instead given by $u\left(  x,y\right)  =\min\left\{
x,y\right\}  $.
\end{itemize}

\item  Solve the problem
\[
\max2x^{\frac{1}{2}}+2y^{\frac{1}{2}}%
\]
subject to the constraints $3x+y\leq10$; $x\geq0$; $y\geq0$ using the method
of Lagrangian multipliers. Write down the Lagrangian function, and each of the
first order conditions before solving the problem.

\item  A consumer has downward sloping indifference curves for goods $x$ and
$y$. Below the 45$^{0}$ line the indifference curves are straight lines with
slope $-\frac{1}{2}$. When the curves hit the 45$^{0}$ line they become
steeper. Above the 45$^{0}$ line they are also straight lines but their slope
is equal to $-2$. Can you use the method we used to prove the existence of a
utility function to provide a utility function that represents these preferences?
\end{enumerate}
\end{document}